\documentclass[]{article}
\usepackage{mathtools}
\usepackage{listings}
\usepackage{enumitem}
\usepackage{amsmath}
\usepackage{amsfonts}
\usepackage{amssymb}
\usepackage[T1]{fontenc} %use different encoding (copy from pdf is now possible}
\usepackage{fullpage} %small margins
\usepackage{color}
\usepackage{graphicx}
\usepackage{float}
\usepackage{calrsfs}
\DeclareMathAlphabet{\pazocal}{OMS}{zplm}{m}{n}
\newcommand{\La}{\mathcal{L}}
\newcommand{\Lb}{\pazocal{L}}

%opening
\title{Languages and Automata Assignment 3}
\author{Dibran Dokter s1047390}

\begin{document}
	
	\maketitle
	
	\section*{1}
	\subsection*{a)}
	\begin{figure}[H]
		\includegraphics[scale=0.3]{/home/owner/git/LanguagesAndAutomata/assignments/week3/images/automata1.jpeg}
	\end{figure}
	This automaton uses the non-determinism to allow for a shorter word, so we can accept 00 and so forth. For the rest we use non-determinism to allow a change between a word with 2's and 1's or 0's. This can be done by taking a path from the top row of states downwards. The automaton does not allow higher numbers in between since there are no transitions the other way around.
	\subsection*{b)}
	In the automaton of 1a 2101 is accepted if we take the path $q_0 \rightarrow q_2 \rightarrow q_6 \rightarrow q_7 \rightarrow q_8$ where $q_8$ is a final state.
	With the parsing of this word we can see the change in the layers of states in action when going from $q_2$ to $q_6$.\\
	
	The word 121 is not accepted since we take the following path: $q_0 \rightarrow q_4$ and then we are unable to parse 2 since there is no transition for it.
	
	\section*{2}
	\subsection*{a)}
	\begin{figure}[H]
		\includegraphics[scale=0.3]{/home/owner/git/LanguagesAndAutomata/assignments/week3/images/automata2.jpeg}
	\end{figure}
	\subsection*{b)}
	\begin{figure}[H]
		\includegraphics[scale=0.3]{/home/owner/git/LanguagesAndAutomata/assignments/week3/images/automata3.jpeg}
	\end{figure}
	When we have multiple initial states (denoted by $I_n$) we can replace those states by the right side as shown. We place a single initial state before the other initial states and add $\lambda$-transitions between the initial state and the $I_n$ states.

	\section*{3}
	\begin{figure}[H]
		\includegraphics[scale=0.3]{/home/owner/git/LanguagesAndAutomata/assignments/week3/images/automata4.jpeg}
	\end{figure}
	All the states from M relate to the same state in M' except for $q_1$. This state is now split between the original state $q_1$ and the new set state $\{q_1,q_2\}$. We do this to make sure that we only get one b transition, to the set state. To make sure that the original automata is reserved we need the a transition back to $q_1$.
	\section*{4}
	\begin{figure}[H]
		\includegraphics[scale=0.3]{/home/owner/git/LanguagesAndAutomata/assignments/week3/images/automata5.jpeg}
	\end{figure}
	To create this automata we can use all the seperate steps to create a NFA-$\lambda$ from a regular expression. We first use the a (for a $\in$ A) step to create the part a. Then we take the part $e_1 \cup e_2$ to take the $b \cup a^{*}$ part. We then take the $a^{*}$ part to make the $a^{*}$ part. In the end we take the last part and connect all the parts using the $e_1e_2$ part.
\end {document}